\apendice{Especificación de Requisitos}

\section{Introducción}
Este anexo aborda los objetivos planteados para este proyecto, identifica los requisitos funcionales establecidos para alcanzar tales metas, y describe los casos de uso asociados a estos requisitos.
\section{Objetivos generales}
\begin{itemize}
    \item Diseñar y desarrollar un motor de físicas que simule un entorno natural simplificado, en dos dimensiones y que ofrezca una interfaz para que se puedan conectar clientes  y poder interactuar con la simulación.
    \item Crear una interfaz gráfica que permita visualizar la simulación y que se pueda interactuar con ella.
\end{itemize}
\section{Catalogo de requisitos}
A continuación, se listan los requisitos específicos que se han definido para el proyecto:
\begin{description}
    \item[RF 1:] El usuario debe poder acceder a un cliente gráfico (web) en el que pueda interactuar con la simulación.
    \begin{description}
        \item[RF 1.1:] El cliente debe ofrecer una sección con una representación gráfica de la simulación.
        \item[RF 1.2:] El cliente debe ofrecer una sección con información específica de los cuerpos.
        \item[RF 1.3:] El cliente debe ofrecer una interfaz para poder interactuar con la simulación.
        \item[RF 1.4:] El programa debe ofrecer un entorno inicial.
        \item[RF 1.5:] El usuario debe poder añadir y eliminar cuerpos a la simulación.
        \item[RF 1.6:] El usuario debe poder seleccionar cuerpos para poder examinarlos mejor.
        \item[RF 1.7:] Una vez existan aeronaves en la simulación y estén seleccionadas, el usuario debe poder establecer un punto del entorno al cual quiere que estas se dirijan.
    \end{description}
    \item[RF 2:] El motor debe ser capaz de simular un entorno natural en dos dimensiones.
    \begin{description}
        \item[RF 2.1:] El entorno debe tener unas dimensiones específicas.
        \item[RF 2.2:] El motor tiene que poder admitir varios tipos distintos de cuerpos.
        \item[RF 2.3:] El motor deberá calcular la posición y comportamiento de los cuerpos en función a unos principios físicos determinados.
        \item[RF 2.4:] El motor debe ser capaz de generar cuerpos en función de unos parámetros.
        \item[RF 2.5:] El motor debe ser capaz de generar un asteroide con un número de vértices aleatorio.
        \item[RF 2.6:] El motor debe ofrecer un método de conectarse con él para obtener el estado de la simulación.
        \item[RF 2.7:] El motor debe ofrecer una interfaz para poder interactuar con la simulación.
    \end{description}
\end{description}
\section{Especificación de requisitos}
\subsection{Diagrama de casos de uso}
\imagen{anexos/usecases.png}{Diagrama de casos de uso del cliente web}{}
\subsection{Actores}
En nuestro sistema, solo existe un actor, el usuario. En caso de que, en un futuro, se implementen nuevas características, será necesario añadir más usuarios (administrador, multiusuario...)
\subsection{Especificación de casos de uso}
\begin{table}[p]
	\centering
	\begin{tabularx}{\linewidth}{ p{0.21\columnwidth} p{0.71\columnwidth} }
		\toprule
		\textbf{CU-1}    & \textbf{Añadir una aeronave}\\
		\toprule
		\textbf{Versión}              & 1.0    \\
		\textbf{Autor}                & Daniel Meruelo Monzón \\
		\textbf{Requisitos asociados} & RF-1.3, RF-1.5, RF-2.2, RF-2.3, RF-2.4, RF-2.6, RF-2.7 \\
		\textbf{Descripción}          & Permite al usuario añadir una aeronave a la simulación. \\
		\textbf{Precondición}         & El motor debe estar arrancado y el cliente web conectado a él. \\
		\textbf{Acciones}             &
		\begin{enumerate}
			\def\labelenumi{\arabic{enumi}.}
			\tightlist
			\item El usuario accede al cliente.
			\item El usuario utiliza el botón `Añadir aeronave`
		\end{enumerate}\\
		\textbf{Postcondición}        & Se crea un nuevo cuerpo del tipo Aeronave \\
		\textbf{Excepciones}          & N/A \\
		\textbf{Importancia}          & Alta\\
		\bottomrule
	\end{tabularx}
	\caption{CU-1 Añadir una aeronave.}
\end{table}
\begin{table}[p]
	\centering
	\begin{tabularx}{\linewidth}{ p{0.21\columnwidth} p{0.71\columnwidth} }
		\toprule
		\textbf{CU-2}    & \textbf{Añadir un planeta}\\
		\toprule
		\textbf{Versión}              & 1.0    \\
		\textbf{Autor}                & Daniel Meruelo Monzón \\
		\textbf{Requisitos asociados} & RF-1.3, RF-1.5, RF-2.2, RF-2.3, RF-2.4, RF-2.6, RF-2.7 \\
		\textbf{Descripción}          & Permite al usuario añadir un planeta a la simulación. \\
		\textbf{Precondición}         & El motor debe estar arrancado y el cliente web conectado a él. \\
		\textbf{Acciones}             &
		\begin{enumerate}
			\def\labelenumi{\arabic{enumi}.}
			\tightlist
			\item El usuario accede al cliente.
			\item El usuario utiliza el botón `Añadir planeta`
		\end{enumerate}\\
		\textbf{Postcondición}        & Se crea un nuevo cuerpo del tipo Planeta \\
		\textbf{Excepciones}          & N/A \\
		\textbf{Importancia}          & Alta\\
		\bottomrule
	\end{tabularx}
	\caption{CU-2 Añadir una planeta.}
\end{table}
\begin{table}[p]
	\centering
	\begin{tabularx}{\linewidth}{ p{0.21\columnwidth} p{0.71\columnwidth} }
		\toprule
		\textbf{CU-3}    & \textbf{Añadir un asteroide}\\
		\toprule
		\textbf{Versión}              & 1.0    \\
		\textbf{Autor}                & Daniel Meruelo Monzón \\
		\textbf{Requisitos asociados} & RF-1.3, RF-1.5, RF-2.2, RF-2.3, RF-2.4, RF 2.5, RF-2.6, RF-2.7 \\
		\textbf{Descripción}          & Permite al usuario añadir un asteroide a la simulación. \\
		\textbf{Precondición}         & El motor debe estar arrancado y el cliente web conectado a él. \\
		\textbf{Acciones}             &
		\begin{enumerate}
			\def\labelenumi{\arabic{enumi}.}
			\tightlist
			\item El usuario accede al cliente.
			\item El usuario utiliza el botón `Añadir asteroide`
		\end{enumerate}\\
		\textbf{Postcondición}        & Se crea un nuevo cuerpo del tipo Asteroide \\
		\textbf{Excepciones}          & N/A \\
		\textbf{Importancia}          & Alta\\
		\bottomrule
	\end{tabularx}
	\caption{CU-3 Añadir una asteroide.}
\end{table}
\begin{table}[p]
	\centering
	\begin{tabularx}{\linewidth}{ p{0.21\columnwidth} p{0.71\columnwidth} }
		\toprule
		\textbf{CU-4}    & \textbf{Añadir un cuerpo aleatorio}\\
		\toprule
		\textbf{Versión}              & 1.0    \\
		\textbf{Autor}                & Daniel Meruelo Monzón \\
		\textbf{Requisitos asociados} & RF-1.3, RF-1.5, RF-2.2, RF-2.3, RF-2.4, RF 2.5, RF-2.6, RF-2.7 \\
		\textbf{Descripción}          & Permite al usuario añadir un cuerpo de tipo aleatorio a la simulación. \\
		\textbf{Precondición}         & El motor debe estar arrancado y el cliente web conectado a él. \\
		\textbf{Acciones}             &
		\begin{enumerate}
			\def\labelenumi{\arabic{enumi}.}
			\tightlist
			\item El usuario accede al cliente.
			\item El usuario utiliza el botón `Añadir cuerpo aleatorio`
		\end{enumerate}\\
		\textbf{Postcondición}        & Se crea un nuevo cuerpo del tipo Asteroide || Planeta || Aeronave\\
		\textbf{Excepciones}          & N/A \\
		\textbf{Importancia}          & Alta\\
		\bottomrule
	\end{tabularx}
	\caption{CU-4 Añadir un cuerpo aleatorio.}
\end{table}
\begin{table}[p]
	\centering
	\begin{tabularx}{\linewidth}{ p{0.21\columnwidth} p{0.71\columnwidth} }
		\toprule
		\textbf{CU-5}    & \textbf{Eliminar un cuerpo}\\
		\toprule
		\textbf{Versión}              & 1.0    \\
		\textbf{Autor}                & Daniel Meruelo Monzón \\
		\textbf{Requisitos asociados} & RF-1.1, RF-1.2, RF-1.3, RF-1.5, RF-1.6, RF-2.2, RF-2.3, RF-2.4, RF-2.6, RF-2.7 \\
		\textbf{Descripción}          & Permite al usuario eliminar un cuerpo de la simulación. \\
		\textbf{Precondición}         & El motor debe estar arrancado y el cliente web conectado a él. Debe existir por lo menos un cuerpo en la simulación. \\
		\textbf{Acciones}             &
		\begin{enumerate}
			\def\labelenumi{\arabic{enumi}.}
			\tightlist
			\item El usuario accede al cliente.
			\item El usuario utiliza el botón `Eliminar` en la tarjeta correspondiente al cuerpo que quiere eliminar.
		\end{enumerate}\\
		\textbf{Postcondición}        & Se elimina el cuerpo seleccionado de la simulación.\\
		\textbf{Excepciones}          & N/A \\
		\textbf{Importancia}          & Alta\\
		\bottomrule
	\end{tabularx}
	\caption{CU-5 Eliminar un cuerpo.}
\end{table}
\begin{table}[p]
	\centering
	\begin{tabularx}{\linewidth}{ p{0.21\columnwidth} p{0.71\columnwidth} }
		\toprule
		\textbf{CU-6}    & \textbf{Seleccionar un cuerpo}\\
		\toprule
		\textbf{Versión}              & 1.0    \\
		\textbf{Autor}                & Daniel Meruelo Monzón \\
		\textbf{Requisitos asociados} & RF-1.1, RF-1.2, RF-1.3, RF-1.5, RF-1.6, RF-2.3, RF-2.7 \\
		\textbf{Descripción}          & Permite al usuario seleccionar un cuerpo de la simulación para mejorar su visibilidad. \\
		\textbf{Precondición}         & El motor debe estar arrancado y el cliente web conectado a él. Debe existir por lo menos un cuerpo en la simulación. \\
		\textbf{Acciones}             &
		\begin{enumerate}
			\def\labelenumi{\arabic{enumi}.}
			\tightlist
			\item El usuario accede al cliente.
			\item El usuario hace click en un cuerpo en la representación gráfica o en la lista de cuerpos.
		\end{enumerate}\\
		\textbf{Postcondición}        & Si el cuerpo no estaba seleccionado, se selecciona. En la representación visual, el cuerpo se dibuja de otro color. En la representación de la información, la tarjeta cambia de color. Si estaba seleccionado, se efectuan los cambios a la inversa.\\
		\textbf{Excepciones}          & N/A \\
		\textbf{Importancia}          & Alta\\
		\bottomrule
	\end{tabularx}
	\caption{CU-6 Seleccionar un cuerpo.}
\end{table}
\begin{table}[p]
	\centering
	\begin{tabularx}{\linewidth}{ p{0.21\columnwidth} p{0.71\columnwidth} }
		\toprule
		\textbf{CU-7}    & \textbf{Establecer objetivo de búsqueda}\\
		\toprule
		\textbf{Versión}              & 1.0    \\
		\textbf{Autor}                & Daniel Meruelo Monzón \\
		\textbf{Requisitos asociados} & RF-1.1, RF-1.2, RF-1.3, RF-1.5, RF-1.6, RF-1.7 RF-2.3, RF-2.6, RF-2.7 \\
		\textbf{Descripción}          & Permite al usuario establecer un objetivo de búsqueda al cual se dirigirán las aeronaves seleccionadas. \\
		\textbf{Precondición}         & El motor debe estar arrancado y el cliente web conectado a él. Debe existir por lo menos un cuerpo de tipo aeronave en la simulación. \\
		\textbf{Acciones}             &
		\begin{enumerate}
			\def\labelenumi{\arabic{enumi}.}
			\tightlist
			\item El usuario accede al cliente.
			\item El usuario hace click en, al menos, un cuerpo del tipo aeronave en la representación gráfica o en la lista de cuerpos.
                \item El cuerpo pasa al estado `seleccionado`.
                \item El usuario hace click en un punto de la representación gráfica de la simulación.
		\end{enumerate}\\
		\textbf{Postcondición}        & Todos los cuerpos del tipo aeronave se dirigirán al punto indicado por el usuario, interrumpiendo sus rutinas habituales.\\
		\textbf{Excepciones}          & N/A \\
		\textbf{Importancia}          & Alta\\
		\bottomrule
	\end{tabularx}
	\caption{CU-7 Establecer un objetivo de búsqueda.}
\end{table}